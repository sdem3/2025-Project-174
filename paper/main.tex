\documentclass{article}
\usepackage{arxiv}

\usepackage[utf8]{inputenc}
\usepackage[english, russian]{babel}
\usepackage[T1]{fontenc}
\usepackage{url}
\usepackage{booktabs}
\usepackage{amsfonts}
\usepackage{nicefrac}
\usepackage{microtype}
\usepackage{lipsum}
\usepackage{lmodern}
\usepackage{graphicx}
\usepackage{natbib}
\usepackage{doi}



\title{Эффекты самоорганизации в рекомендательных системах}

\author{ Дементьев Сергей \\
        МФТИ\\
	\texttt{dementev.sa@phystech.edu}  \\
	\And
	Веприков Андрей \\
        МФТИ \\
	\texttt{veprikov.as@phystech.edu}  \\
	\And
    Хританков Антон \\
    ВШЭ, МФТИ\\
     \texttt{akhritankov@hse.ru} \\
}
\date{}

\renewcommand{\undertitle}{}
\renewcommand{\shorttitle}{Эффекты самоорганизации в рекомендательных системах}

%%% does not show
%%% DeclareMathOperator*{\norm}{norm}

%%% Add PDF metadata to help others organize their library
%%% Once the PDF is generated, you can check the metadata with
%%% $ pdfinfo template.pdf
%%% \hypersetup{
%%% pdftitle={A template for the arxiv style},
%%% pdfsubject={q-bio.NC, q-bio.QM},
%%% pdfauthor={David S.~Hippocampus, Elias D.~Striatum},
%%% pdfkeywords={First keyword, Second keyword, More},
%%% }

\begin{document}
\maketitle

\begin{abstract}
    В работе исследуются петли скрытой обратной связи в рекомендательных системах.
    Решается задача поиска условий возникновения положительной обратной связи. Исследуется эффект самоорганизации в рекомендательной системе, в которой "товары" и "пользователи" меняются со временем.

\end{abstract}

\keywords{Петли обратной связи \and Рекомендательные системы \and Контролируемое машинное обучение}

\section{Введение}
TODO

\section{Постановка задачи}

TODO

\section{Метод}

TODO

\section{Теория}

TODO


\section{Вычислительный эксперимент}

TODO

\subsection{Описание данных}

TODO


\subsection{Модель}

TODO


\subsection{Результаты}
TODO

\section{Заключение}

TODO

\bibliographystyle{plain}
\bibliography{Gorbulev2023TopicModels}

\end{document}